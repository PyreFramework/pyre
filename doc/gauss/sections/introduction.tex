% -*- LaTeX -*-
% -*- coding: utf-8 -*-
%
% michael a.g. aïvázis
% orthologue
% (c) 1998-2021 all rights reserved
%

\section{Introduction}
\label{sec:introduction}

Research codes are peculiar beasts. They are typically born to parents that are far too young
and inexperienced to take proper care of them. Their early days are sickly and even their
authors do not expect them to live any longer than the next paper, definitely not past the
completion of their thesis. They grow up in haphazard ways, reflecting the evolution of their
care provider's understanding of some research problem, end up having far too many appendages
sticking out of all the wrong places, and display very few signs of any organizing principle,
let alone design.  Yet, many of them outlive their parent's wildest expectations (or fears),
have long, productive lives, and become focal points of entire research communities. Hated by
all, but also used by all.

The good ones have buried in them precious intellectual capital; that's the secret of their
longevity. The reason they are constantly on the some-day-I-will-rewrite-this list\footnote{a
  list that invariably matures into {\em someone-else}-should-some-day-rewrite-this} is almost
always the lack of enough structure so that successive generations of foster parents can
maintain and evolve the code. The missing structure goes by the name ``modern software
engineering practices'' and it's not on the list of skills that graduate students of
respectable institutions are supposed to have.

Pyre, the software architecture described in this paper, is an attempt to bring state of the
art software design practices to scientific computing. The goal is to provide a strong skeleton
on which to build scientific codes by steering the implementation towards usability and
maintainability. It's not a substitute for the intelligence, experience and effort necessary to
write robust, hardened software. But by encouraging you to ask the right design questions and
make the right practices part of your software cycle, you should experience a dramatic
improvement in the quality of code you write.

You will still need to shop around for a source control system and find a scalable way to build
your software on multiple platforms. You will need to find a good solution for writing
documentation, maintain and run test suites, and track bugs and feature requests. If you are
really ambitious, you need a release management solution. Most people hope they can get away
without all this overhead, but that's just the mild form of delusion that comes from not
knowing what you are in for.  Writing software can easily degenerate into a chaotic practice,
with small changes having potentially unbounded effects, even when it's only you that's doing
the coding. The passage of time has a way of introducing interesting complexity in software
systems.

In the next few sections, I will show you how to turn a throw-away script into a usable
application. After a brief introduction to the method, we will start by writing a na\"ive
implementation of Monte Carlo integration in both python and \cpp. Then, we will evolve the
code by introducing object oriented concepts that will help us improve--and document--the
design of the code. The last evolutionary step will be to cast the design as a collection of
reusable components. Once we have that, we will explore user interfaces, parallelism and more.

% end of file
