% -*- LaTeX -*-
% -*- coding: utf-8 -*-
%
% michael a.g. aïvázis
% orthologue
% (c) 1998-2020 all rights reserved
%

% --------------------------------------

\section{montecarlo}
\subsection{theory}

% --------------------------------------
% setting up the problem
\begin{frame}
%
  \frametitle{Monte Carlo integration}
%
  \begin{itemize}
%
  \item let $f$ be sufficiently well behaved in a region $\Omega \subset \mathbb{R}^{n}$ and
    consider the integral
    \begin{equation}
      I_{\Omega} (f) = \int_{\Omega} f
      \label{eq:integral}
    \end{equation}
%
  \item the {\em Monte Carlo} method approximates the value of the integral in \eqref{integral}
    by sampling $f$ at random points in $\Omega$
%
  \item let $X_{N}$ be such a sample of $N$ points; then the Monte Carlo estimate is given by
    \begin{equation}
      I_{\Omega} (f; X_{N})
      =
      \Omega \cdot \langle f \rangle
      =
      \Omega \, \frac{1}{N} \sum_{x \in X_{N}} f(x)
      \label{eq:mc-estimate}
    \end{equation}
%
    where $\langle f \rangle$ is the sample mean of $f$, and $\Omega$ is used as a shorthand
    for the volume of the integration region. Details in \citep{hammersley,ueberhuber}; see
    \cite{weinzierl} for an excellent pedagogical introduction.
%
  \item the approximation error falls like $1/\sqrt{N}$
    \begin{itemize}
    \item rather slow
    \item but dimension independent!
    \end{itemize}
  \end{itemize}
%
\end{frame}

% --------------------------------------
% mc implementation
\begin{frame}
%
  \frametitle{Implementation strategy}
%
  \begin{itemize}
%
  \item computer implementations require a pseudo-random number generator to build the sample
%
    \item most generators return numbers in $(0,1)$ so
      \begin{itemize}
      \item find a box $B$ that contains $\Omega$
      \item generate $n$ numbers to build a point in the unit $\mathbb{R}^{n}$ cube
      \item stretch and translate the unit cube onto $B$
      \end{itemize}
%
  \item the integration is restricted to $\Omega$ by introducing
    \begin{equation}
        \Theta_{\Omega}
        =
        \left\{
          \begin{array}{ll}
            1 & x \in \Omega \\
            0 & {\rm otherwise}
          \end{array}
        \right.
        \label{eq:theta}
    \end{equation}
    to get
    \begin{equation}
      I_{\Omega} (f)
      =
      \int_{B} \Theta_{\Omega} \, f
      \label{eq:integral-box}
    \end{equation}
%
  \end{itemize}
%
\end{frame}

% --------------------------------------
% reorganize to a form better suited for implementation
\begin{frame}
%
  \frametitle{Recasting Monte Carlo integration}
%
  \begin{itemize}
%
  \item there are now two classes of points in the sample $X_{N}$
    \begin{itemize}
    \item those in $\Omega$
    \item and the rest
    \end{itemize}
%
  \item let $\tilde{N}$ be the number of sample points in $\Omega$; \eqref{mc-estimate} becomes
    \begin{equation}
      I_{\Omega} (f; X_{N})
      =
      \Omega \, \frac{1}{\tilde{N}} \sum_{x \in X_{\tilde{N}}} f(x)
      \label{eq:mc-box}
    \end{equation}
%
  \item let $B$ be the volume of the sampling box; observe that the volume of the integration
    region can be approximated by
    \begin{equation}
      \Omega = \frac{\tilde{N}}{N} B
    \end{equation}
%
    and the sum over the points $x \in X_{\tilde{N}}$ can be extended to the entire sample
    $X_{N}$ by using the filter $\Theta_{\Omega}$
%
    \begin{equation}
      I_{\Omega} (f; X_{N})
      =
      B \, \frac{1}{N} \sum_{x \in X_{N}} \Theta_{\Omega} \, f(x)
    \end{equation}
%
  \end{itemize}
%
\end{frame}


% --------------------------------------
% recap
\begin{frame}
%
  \frametitle{Requirements}
%
  \begin{itemize}
%
  \item to summarize, the Monte Carlo approximation is computed from
%
    \begin{equation}
      I_{\Omega} (f; X_{N})
      =
      B \, \frac{1}{N} \sum_{x \in X_{N}} \Theta_{\Omega} \, f(x)
    \end{equation}
%
    using
    \begin{itemize}
    \item an implementation of the function $f$ to be integrated over $\Omega$
    \item an $n$-dimensional box $B$ that contains $\Omega$
    \item a good pseudo-random number generator to build the sample $X_{N} \in B$
    \item a routine to test points $x \in X_{N}$ and return \keyword{false} if they are
      exterior to $\Omega$ and \keyword{true} otherwise
    \end{itemize}
%
    to sum the values of the integrand on points interior to $\Omega$, and scale by the
    volume of the bounding box $B$ over the sample size $N$
%
  \item essentially a reduction
    \begin{itemize}
    \item should be straightforward to implement in parallel
    \end{itemize}
%
  \item rich enough structure to be a non-trivial \pyre\ application
%
  \end{itemize}
%
\end{frame}

% --------------------------------------

\subsection{example}

% --------------------------------------
% preview: a python script
\begin{frame}
%
  \frametitle{A trivial script}
%
  \begin{itemize}
  \item estimating $\pi$ using Monte Carlo integration over a quarter disk
  \end{itemize}
%
  \python{
    firstnumber=9,linerange={9-25}, xleftmargin=4em,
    label={lst:python:pi},
    caption={\srcfile{pi.py}: Estimating $\pi$ in python},
  }{listings/pi.py}

%
\end{frame}

% end of file
